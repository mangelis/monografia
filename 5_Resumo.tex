%%%%%%%%%%%%%%%%%%%%%%%%%%%%%%%%%%%%%%%%%%%%%%%%%%%%%%%%%%%%%%%%%%%%%%%%%%%%%%%%%%%%%%
% ---------------------------------------------------------------------------------  %
% ---------------------------------------------------------------------------------  %
%                                                                                    %
%                  MODELO DE MONOGRAFIA DO E-COMP - POLI - UPE                       %
%                                                                                    %
% ---------------------------------------------------------------------------------  %
% ---------------------------------------------------------------------------------  %
%%%%%%%%%%%%%%%%%%%%%%%%%%%%%%%%%%%%%%%%%%%%%%%%%%%%%%%%%%%%%%%%%%%%%%%%%%%%%%%%%%%%%%

\setlength{\absparsep}{18pt} % ajusta o espa�amento dos par�grafos do resumo

\begin{resumo}
    \begin{center}
        \begin{minipage}{0.7\textwidth}

            \noindent O resumo � uma apresenta��o do trabalho na l�ngua original em que o mesmo foi escrito, redigido de forma objetiva e com destaque dos principais elementos do conte�do pesquisado. Na confec��o do resumo devem ser observados os seguintes aspectos:

            \begin{itemize}
                \item Ressaltar os objetivos, os m�todos, os resultados e as conclus�es do trabalho;
                \item Deve ser escrito em um �nico par�grafo;
                \item Utilizar, no m�ximo, 250 palavras no texto do resumo.
            \end{itemize}

            \noindent \textbf{Palavras-chave}: Redes Complexas, Classfica��o textual, Processamento de texto, Estilometria, Reconhecimento de padr�es, Perceptron de m�ltiplas camadas.
        \end{minipage}
    \end{center}
\end{resumo} 