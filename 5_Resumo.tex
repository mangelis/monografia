%%%%%%%%%%%%%%%%%%%%%%%%%%%%%%%%%%%%%%%%%%%%%%%%%%%%%%%%%%%%%%%%%%%%%%%%%%%%%%%%%%%%%%
% ---------------------------------------------------------------------------------  %
% ---------------------------------------------------------------------------------  %
%                                                                                    %
%                  MODELO DE MONOGRAFIA DO E-COMP - POLI - UPE                       %
%                                                                                    %
% ---------------------------------------------------------------------------------  %
% ---------------------------------------------------------------------------------  %
%%%%%%%%%%%%%%%%%%%%%%%%%%%%%%%%%%%%%%%%%%%%%%%%%%%%%%%%%%%%%%%%%%%%%%%%%%%%%%%%%%%%%%

\setlength{\absparsep}{18pt} % ajusta o espa�amento dos par�grafos do resumo

\begin{resumo}
    \begin{center}
        \begin{minipage}{0.7\textwidth}
			 \noindent
             A estilografia e o reconhecimento de autoria vem sendo frequentemente objeto de estudo de v�rios pesquisadores ao redor do mundo. Por tratarem os textos majoritariamente por suas propriedades estat�sticas, sofreram grande avan�o nos �ltimos anos em conjunto com as t�cnicas de apredizagem de m�quina. Paralelamente, uma nova abordagem para representar e modelar sistemas complexos tem ganhado for�a: as redes complexas. Elas t�m modelado muito satisfatoriamente v�rios sistemas reais, entre eles, textos. A rede textual mais utilizada para reconhecimento de autoria � a de co-ocorr�ncia de palavras. Neste trabalho � abordada uma perspectiva n�o presente na literatura para criar redes a partir de textos. Para que essas redes possam ser criadas, v�rias etapas de pr�-processamento s�o requeridas. Entre elas, podem ser citadas a segmenta��o de senten�as e o \emph{part-of-speech tagging}. Pares de classes gramaticais s�o associados a outros pares vizinhos, criando um grafo de co-ocorr�ncia desses pares. V�rias m�tricas de redes complexas s�o extra�das do grafo que representa a rede, e desses valores s�o extra�das medidas que tentam representar as caracter�sticas estil�sticas do autor. O conjunto de caracter�sticas � apresentado a um classificador que reconhecer� quem escreveu um certo texto dado. Os resultados foram interessantes e sugeriram que palavras comumente n�o utilizadas para o reconhecimento de autoria, na verdade, podem ser bastante �teis quando aplicadas �s redes geradas neste trabalho.

            \noindent \textbf{Palavras-chave}: Redes Complexas, Classfica��o textual, Processamento de texto, Estilometria, Reconhecimento de padr�es, Perceptron de m�ltiplas camadas.
        \end{minipage}
    \end{center}
\end{resumo} 