%%%%%%%%%%%%%%%%%%%%%%%%%%%%%%%%%%%%%%%%%%%%%%%%%%%%%%%%%%%%%%%%%%%%%%%%%%%%%%%%%%%%%%
% ---------------------------------------------------------------------------------  %
% ---------------------------------------------------------------------------------  %
%                                                                                    %
%                  MODELO DE MONOGRAFIA DO E-COMP - POLI - UPE                       %
%                                                                                    %
% ---------------------------------------------------------------------------------  %
% ---------------------------------------------------------------------------------  %
%%%%%%%%%%%%%%%%%%%%%%%%%%%%%%%%%%%%%%%%%%%%%%%%%%%%%%%%%%%%%%%%%%%%%%%%%%%%%%%%%%%%%%

\chapter{Referencial Te�rico \label{cap:referencial} }
\section{Grafos}
Um grafo $\mathcal{G=(V,E)}$ � definido por um conjunto $\mathcal{V=}$$\{v_1,v_2, \ldots,v_n\}$ de v�rtices 
e por um conjunto $\mathcal{E=}$$\{e_1,e_2, \ldots,e_m\}$ de arestas tais que $\mathcal{E \subseteq V \times V }$
e $\mathcal{V \cap E = \emptyset}$. Eles podem ser direcionados ou n�o. No primeiro caso, suas arestas s�o pares
ordenados de v�rtices, j� no segundo, a ordem � indiferente. Al�m disso, valores num�ricos podem ser associados
a cada aresta. Se este for o caso, o grafo � denominado ponderado. Um caminho de comprimento $L$ � uma sequ�ncia
alternada $v_{1}e_{1}v_{2}e_{2}v_{3} \ldots v_{L-1}e_{L-1}v_{L}$ de v�rtices $v_{i}$ e arestas $e_{i}$ de forma que
o v�rtice de destino de $e_{i}$ � $v_{i}$, e seu v�rtice de destino, $v_{i+1}$. Um c�clo � um caso particular de caminho onde $v_{1}=v_{L}$. Diz-se que um grafo � conectado quando existe ao menos um caminho conectando todos os pares de v�rtices.

\par
Um grafo $\mathcal{G}$ com $n$ v�rtices pode ser representado por uma matriz $\mathcal{A}=\left[a_{ij}\right]$ $n \times n$ como segue:

\begin{gather}
	a_{ij} =
		\begin{cases}
			w_{ij} & \text{se existe uma aresta de } a_{i} \text{ para } a_{j} \\
			0 & \text{caso contr�rio }
		\end{cases} 
	\intertext{Onde:}
	\begin{tabular}{>{$}r<{$}@{\ :\ }l}
		w_{ij} & � o peso da aresta entre os v�rtices $v_{i}$ e $v_{j}$
	\end{tabular}\nonumber
\end{gather}

\section{Redes Complexas}

\section{Processamento de Texto}

\section{Tipos de Redes Textuais}

\section{Multilayer Perceptron}
\subsection{Conceito 1}
\subsection{Conceito 2}
\section{Trabalhos Relacionados}