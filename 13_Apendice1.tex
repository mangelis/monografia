%%%%%%%%%%%%%%%%%%%%%%%%%%%%%%%%%%%%%%%%%%%%%%%%%%%%%%%%%%%%%%%%%%%%%%%%%%%%%%%%%%%%%%
% ---------------------------------------------------------------------------------  %
% ---------------------------------------------------------------------------------  %
%                                                                                    %
%                  MODELO DE MONOGRAFIA DO E-COMP - POLI - UPE                       %
%                                                                                    %
% ---------------------------------------------------------------------------------  %
% ---------------------------------------------------------------------------------  %
%%%%%%%%%%%%%%%%%%%%%%%%%%%%%%%%%%%%%%%%%%%%%%%%%%%%%%%%%%%%%%%%%%%%%%%%%%%%%%%%%%%%%%

\chapter{Tabela de \emph{part-of-speech tags}}
\label{apendice:tabela_pos}
A tabela a seguir cont�m a lista de \emph{tags} utilizadas pelo \emph{Stanford part-of-speech tagger:} 

\begin{center}
	\newcolumntype{Y}{>{\centering\arraybackslash}c}
	\begin{longtable}{|c|c|p{6cm}|}
		\caption{Lista de \emph{tags} utilizadas pelo \emph{Stanford part-of-speech tagger}} \label{tabela:pos_tags} \\
		
		\hline \multicolumn{1}{|c|}{\textbf{\emph{Tag}}} & \multicolumn{1}{c|}{\textbf{Descri��o}} & \multicolumn{1}{c|}{\textbf{Exemplos}} \\ \hline
		\endfirsthead
		
		\multicolumn{3}{c}%
		{{\bfseries \tablename\ \thetable{} -- continua��o da p�gina anterior}} \\
		\hline \multicolumn{1}{|c|}{\textbf{\emph{Tag}}} &
		\multicolumn{1}{c|}{\textbf{Descri��o}} &
		\multicolumn{1}{c|}{\textbf{Exemplos}} \\ \hline
		\endhead
		
		\hline \multicolumn{3}{|r|}{{Continua��o na pr�xima p�gina}} \\ \hline
		\endfoot
		
		\hline \hline
		\endlastfoot
		
		\$ & d�lar & \$ -\$ --\$ A\$ C\$ HK\$ M\$ NZ\$ S\$ U.S.\$ US\$ \\ \hline
		`` & in�cio de cita��o & ` `` \\
		\hline
		'' & fim de cita��o & ' '' \\
		\hline
		( & par�nteses � esquerda & ( [ \{ \\
		\hline
		) & par�nteses � direita & ) ] \} \\
		\hline
		, & v�rgula & , \\
		\hline
		-- & trevess�o & -- \\
		\hline
		. & delimitador de frase & . ! ? \\
		\hline
		; & ponto e v�rgula ou elipse & : ; $\dots$ \\
		\hline
		CC & conjun��o coordenada & \& 'n and both but either et for less minus neither nor or plus so
		therefore times v. versus vs. whether yet \\
		\hline
		CD & numeral cardinal & mid-1890 nine-thirty forty-two one-tenth ten million 0.5 one forty-seven 1987 twenty '79 zero two 78-degrees eighty-four IX '60s .025 fifteen 271,124 dozen quintillion DM2,000 \\
		\hline
		DT & determinador & all an another any both del each either every half la many much nary neither no some such that the them these this those \\
		\hline
		EX & \emph{there} existencial & there \\
		\hline
		FW & palavra estrangeira & gemeinschaft hund ich jeux habeas Haementeria Herr K'ang-si vous lutihaw alai je jour objets salutaris fille quibusdam pas trop Monte terram fiche oui corporis \\
		\hline
		IN & preposi��o ou conjun��o subordinada & astride among uppon whether out inside pro despite on by throughout below within for towards near behind atop around if like until below next into if beside \\
		\hline
		JJ & adjetivo ou numeral ordinal & third ill-mannered pre-war regrettable oiled calamitous first separable ectoplasmic battery-powered participatory fourth still-to-be-named multilingual multi-disciplinary \\
		\hline
		JJR & adjetivo comparativo & thirdbleaker braver breezier briefer brighter brisker broader bumper busier calmer cheaper choosier cleaner clearer closer colder commoner costlier cozier creamier crunchier cuter \\
		\hline
		JJS & adjetivo superlativo & calmest cheapest choicest classiest cleanest clearest closest commonest corniest costliest crassest creepiest crudest cutest darkest deadliest dearest deepest densest dinkiest  \\
		\hline
		LS & marcador de itens em listas & A A. B B. C C. D E F First G H I J K One SP-44001 SP-44002 SP-44005 SP-44007 Second Third Three Two \* a b c d first five four one six three two  \\
		\hline
		MD & verbo modal auxiliar & can cannot could couldn't dare may might must need ought shall should shouldn't will would  \\
		\hline
		NN & nome comum, singular ou\emph{mass} (incont�vel)  & common-carrier cabbage knuckle-duster Casino afghan shed thermostat investment slide humour falloff slick wind hyena override subhumanity machinist blood  \\
		\hline
		NNP & nome pr�prio, singular  & Motown Venneboerger Czestochwa Ranzer Conchita Trumplane Christos Oceanside Escobar Kreisler Sawyer Cougar Yvette Ervin ODI Darryl CTCA Shannon A.K.C. Meltex Liverpool  \\
		\hline
		NNPS & nome pr�prio, plural  & Americans Americas Amharas Amityvilles Amusements Anarcho-Syndicalists Andalusians Andes Andruses Angels Animals Anthony Antilles Antiques Apache Apaches Apocrypha  \\
		\hline
		NNS & nome comum, plural  & undergraduates scotches bric-a-brac products bodyguards facets coasts divestitures storehouses designs clubs fragrances averages subjectivists apprehensions muses factory-jobs  \\
		\hline
		PDT & pr�-delimitador  & all both half many quite such sure this  \\
		\hline
		POS & indicador genitivo  & ' 's  \\
		\hline
		PRP & pronome pessoal  & hers herself him himself hisself it itself me myself one oneself ours ourselves ownself self she thee theirs them themselves they thou thy us  \\
		\hline
	\end{longtable}
\end{center}
