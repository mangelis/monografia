%%%%%%%%%%%%%%%%%%%%%%%%%%%%%%%%%%%%%%%%%%%%%%%%%%%%%%%%%%%%%%%%%%%%%%%%%%%%%%%%%%%%%%
% ---------------------------------------------------------------------------------  %
% ---------------------------------------------------------------------------------  %
%                                                                                    %
%                  MODELO DE MONOGRAFIA DO E-COMP - POLI - UPE                       %
%                                                                                    %
% ---------------------------------------------------------------------------------  %
% ---------------------------------------------------------------------------------  %
%%%%%%%%%%%%%%%%%%%%%%%%%%%%%%%%%%%%%%%%%%%%%%%%%%%%%%%%%%%%%%%%%%%%%%%%%%%%%%%%%%%%%%

\setlength{\absparsep}{18pt} % ajusta o espa�amento dos par�grafos do resumo

\begin{resumo}[Abstract]
	\begin{otherlanguage*}{english}
	    \begin{center}
	        \begin{minipage}{0.7\textwidth}
				\noindent O abstract � igual ao resumo na l�ngua vern�cula, com a diferen�a de que vem escrito em uma l�ngua estrangeira. Usualmente, � feito em ingl�s, mas tamb�m pode ser em espanhol ou franc�s. Para o modelo aqui descrito, ser� adotada a l�ngua inglesa.
		            
				\noindent \textbf{Keywords}: word1, word2, etc.            
	        \end{minipage}
	    \end{center}
	\end{otherlanguage*}
\end{resumo}
