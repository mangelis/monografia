%%%%%%%%%%%%%%%%%%%%%%%%%%%%%%%%%%%%%%%%%%%%%%%%%%%%%%%%%%%%%%%%%%%%%%%%%%%%%%%%%%%%%%
% ---------------------------------------------------------------------------------  %
% ---------------------------------------------------------------------------------  %
%                                                                                    %
%                  MODELO DE MONOGRAFIA DO E-COMP - POLI - UPE                       %
%                                                                                    %
% ---------------------------------------------------------------------------------  %
% ---------------------------------------------------------------------------------  %
%%%%%%%%%%%%%%%%%%%%%%%%%%%%%%%%%%%%%%%%%%%%%%%%%%%%%%%%%%%%%%%%%%%%%%%%%%%%%%%%%%%%%%

\setlength{\absparsep}{18pt} % ajusta o espa�amento dos par�grafos do resumo

\begin{resumo}[Abstract]
	\begin{otherlanguage*}{english}
	    \begin{center}
	        \begin{minipage}{0.7\textwidth}
				\noindent 
		         The stylography and authorship recognition have been frequently studied by many researchers around the world.
		         By treating the texts mainly by its statistical properties, these fields suffered major advance in recent years. 
		         Parallely, a new approach to represent and model complex systems have got stronger: the complex networks.
		         They have satisfactorily modelled several real systems, among them, texts. The most used text network in authorship recognition is the co-occurrence of words. In this work, a new perspective to create networks from texts is addressed.
		         For this networks to be built, many pre-processing steps are required. For example, sentence segmentation and part-of-speech tagging. Part-of-speech pairs are associated to other pairs, this builds a co-occurrence graph of these pairs.
		         Several complex networks metrics are extracted from the graph that represents the networks. From these values graph characterizing measures are obtained which can represent the author's stylistic properties. The set of features is presented to a classifier and it will recognize who wrote that text. There are interesting results in this work. They suggest that commonly rejected words (known as stopwords)  are actually useful for the networks proposed by this work. \\
				\noindent \textbf{Keywords}: Complex Networks, Text Classification, Text Processing, Stylometry, Pattern Recognition, Multilayer Perceptron.            
	        \end{minipage}
	    \end{center}
	\end{otherlanguage*}
\end{resumo}
